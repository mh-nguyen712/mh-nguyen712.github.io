\documentclass[10pt,A4]{article}	
\usepackage[utf8]{inputenc}		
\usepackage{xstring, xifthen}

%----------------------------------------------------------------------------------------
%	FONT BASICS
%----------------------------------------------------------------------------------------

% some tex-live fonts - choose your own

%\usepackage[defaultsans]{droidsans}
%\usepackage[default]{comfortaa}
%\usepackage{cmbright}
\usepackage[default]{raleway}
%\usepackage{fetamont}
%\usepackage[default]{gillius}
%\usepackage[light,math]{iwona}
%\usepackage[thin]{roboto} 
\usepackage{float}

% set font default
\renewcommand*\familydefault{\sfdefault} 	
\usepackage[T1]{fontenc}

% more font size definitions
\usepackage{moresize}

%----------------------------------------------------------------------------------------
%	FONT AWESOME ICONS
%---------------------------------------------------------------------------------------- 

% include the fontawesome icon set
\usepackage{fontawesome}

% use to vertically center content
% credits to: http://tex.stackexchange.com/questions/7219/how-to-vertically-center-two-images-next-to-each-other
\newcommand{\vcenteredinclude}[1]{\begingroup
\setbox0=\hbox{\includegraphics{#1}}%
\parbox{\wd0}{\box0}\endgroup}

% use to vertically center content
% credits to: http://tex.stackexchange.com/questions/7219/how-to-vertically-center-two-images-next-to-each-other
\newcommand*{\vcenteredhbox}[1]{\begingroup
\setbox0=\hbox{#1}\parbox{\wd0}{\box0}\endgroup}

% icon shortcut
\newcommand{\icon}[3] { 							
	\makebox(#2, #2){\textcolor{maincol}{\csname fa#1\endcsname}}
}	

% icon with text shortcut
\newcommand{\icontext}[4]{ 						
	\vcenteredhbox{\icon{#1}{#2}{#3}}  \hspace{2pt}  \parbox{0.9\mpwidth}{\textcolor{#4}{#3}}
}

% icon with website url
\newcommand{\iconhref}[5]{ 						
    \vcenteredhbox{\icon{#1}{#2}{#5}}  \hspace{2pt} \href{#4}{\textcolor{#5}{#3}}
}

% icon with email link
\newcommand{\iconemail}[5]{ 						
    \vcenteredhbox{\icon{#1}{#2}{#5}}  \hspace{2pt} \href{mailto:#4}{\textcolor{#5}{#3}}
}

%----------------------------------------------------------------------------------------
%	PAGE LAYOUT  DEFINITIONS
%----------------------------------------------------------------------------------------

% page outer frames (debug-only)
% \usepackage{showframe}		

% we use paracol to display breakable two columns
\usepackage{paracol}

% define page styles using geometry
\usepackage[a4paper]{geometry}

% remove all possible margins
\geometry{top=1.5cm, bottom=1.5cm, left=1.5cm, right=1.5cm}

\usepackage{fancyhdr}
\pagestyle{empty}

% space between header and content
% \setlength{\headheight}{0pt}

% indentation is zero
\setlength{\parindent}{0mm}

%----------------------------------------------------------------------------------------
%	TABLE /ARRAY DEFINITIONS
%---------------------------------------------------------------------------------------- 

% extended aligning of tabular cells
\usepackage{array}

% custom column right-align with fixed width
% use like p{size} but via x{size}
\newcolumntype{x}[1]{%
>{\raggedleft\hspace{0pt}}p{#1}}%

% 
% FOR PUBLICATIONS
% \usepackage[backend=bibtex,style=numeric]{biblatex}
\usepackage[backend=biber,
           bibstyle=numeric,sorting=ydnt,sortcites=true,natbib=true,defernumbers=true,
           maxbibnames=99,giveninits=true,uniquename=false]{biblatex}
\addbibresource{papers.bib}

%----------------------------------------------------------------------------------------
%	GRAPHICS DEFINITIONS
%---------------------------------------------------------------------------------------- 

%for header image
\usepackage{graphicx}

% use this for floating figures
% \usepackage{wrapfig}
% \usepackage{float}
% \floatstyle{boxed} 
% \restylefloat{figure}

%for drawing graphics		
\usepackage{tikz}				
\usetikzlibrary{shapes, backgrounds,mindmap, trees}

%----------------------------------------------------------------------------------------
%	Color DEFINITIONS
%---------------------------------------------------------------------------------------- 
\usepackage{transparent}
\usepackage{color}

% \definecolor{babyblue}{rgb}{0.54, 0.81, 0.94}
% primary color
\definecolor{maincol}{rgb}{0.54, 0.81, 0.94}

% accent color, secondary
% \definecolor{accentcol}{RGB}{ 250, 150, 10 }

% dark color
\definecolor{darkcol}{rgb}{0.36, 0.54, 0.66}

% light color
\definecolor{lightcol}{RGB}{245,245,245}


% Package for links, must be the last package used
\usepackage[hidelinks]{hyperref}

% returns minipage width minus two times \fboxsep
% to keep padding included in width calculations
% can also be used for other boxes / environments
\newcommand{\mpwidth}{\linewidth-\fboxsep-\fboxsep}
	


%============================================================================%
%
%	CV COMMANDS
%
%============================================================================%

%----------------------------------------------------------------------------------------
%	 CV LIST
%----------------------------------------------------------------------------------------

% renders a standard latex list but abstracts away the environment definition (begin/end)
\newcommand{\cvlist}[1] {
	\begin{itemize}{#1}\end{itemize}
}

%----------------------------------------------------------------------------------------
%	 CV TEXT
%----------------------------------------------------------------------------------------

% base class to wrap any text based stuff here. Renders like a paragraph.
% Allows complex commands to be passed, too.
% param 1: *any
\newcommand{\cvtext}[1] {
	\begin{tabular*}{\linewidth}{p{1\mpwidth}}
		\parbox{1\mpwidth}{#1}
	\end{tabular*}
}

%----------------------------------------------------------------------------------------
%	CV SECTION
%----------------------------------------------------------------------------------------

% Renders a a CV section headline with a nice underline in main color.
% param 1: section title
\newcommand{\cvsection}[1] {
	\vspace{7pt}
	% \cvtext{
		\textbf{\large{\textcolor{darkcol}{\uppercase{#1}}}}\\[-6pt]
		\textcolor{maincol}{ \rule{\textwidth}{0.75pt} } %\vspace{0.1cm}
	% }
}

%----------------------------------------------------------------------------------------
%	META SKILL
%----------------------------------------------------------------------------------------

% Renders a progress-bar to indicate a certain skill in percent.
% param 1: name of the skill / tech / etc.
% param 2: level (for example in years)
% param 3: percent, values range from 0 to 1
\newcommand{\cvskill}[3] {
	\begin{tabular*}{1\mpwidth}{p{0.72\mpwidth}  r}
 		\textcolor{black}{\textbf{#1}} & \textcolor{maincol}{#2}\\
	\end{tabular*}%
	
	\hspace{4pt}
	\begin{tikzpicture}[scale=1,rounded corners=2pt,very thin]
		\fill [lightcol] (0,0) rectangle (1\mpwidth, 0.15);
		\fill [maincol] (0,0) rectangle (#3\mpwidth, 0.15);
  	\end{tikzpicture}%
}

\newcommand{\skill}[1] {
    \vspace{0.2cm}
	\begin{tabular*}{1\mpwidth}{p{0.9\mpwidth}  r}
 		\textcolor{black}{\textbf{#1}} \\
	\end{tabular*}%
}

\newcommand{\technique}[2] {
    \vspace{0.2cm}
	\begin{tabular*}{1\mpwidth}{1\mpwidth}
 		\textcolor{black}{\textbf{#1}} & \textcolor{darkcol}{#2}\\
	\end{tabular*}%
}

\newcommand{\cvskilllange}[3] {
	\begin{tabular*}{1\mpwidth}{p{0.6\mpwidth}  r}
 		\textcolor{black}{\textbf{#1}} & \textcolor{maincol}{#2}\\
	\end{tabular*}%
	
	\hspace{4pt}
	\begin{tikzpicture}[scale=1,rounded corners=2pt,very thin]
		\fill [lightcol] (0,0) rectangle (1\mpwidth, 0.15);
		\fill [maincol] (0,0) rectangle (#3\mpwidth, 0.15);
  	\end{tikzpicture}%
}


%----------------------------------------------------------------------------------------
%	 CV EVENT
%----------------------------------------------------------------------------------------
\newcommand{\cvevent}[7] {
	% we wrap this part in a parbox, so title and description are not separated on a pagebreak
	% if you need more control on page breaks, remove the parbox
% 	\parbox{\mpwidth}{
		\begin{tabular*}{1\mpwidth}{p{0.72\mpwidth}  r}
	 		\textcolor{black}{\textbf{#2}} & \colorbox{maincol}{\makebox[0.25\mpwidth]{\textcolor{white}{#1}}}\\
			\textcolor{maincol}{\textbf{#3}} & 
		\end{tabular*} %\\ %[2pt]
	
		\ifthenelse{\isempty{#4}}{}{
			\cvtext{#4}\\
		}
% 	}

	\ifthenelse{\isempty{#5}}{}{
		\vspace{2pt}
		{#5}
	}
	\vspace{2pt}
}

%----------------------------------------------------------------------------------------
%	 CV META EVENT
%----------------------------------------------------------------------------------------

% Renders a CV event on the sidebar
% param 1: title
% param 2: subtitle (optional)
% param 3: customer, employer, etc,. (optional)
% param 4: info text (optional)
\newcommand{\cvmetaevent}[4] {
	\textcolor{maincol} {\cvtext{\textbf{\begin{flushleft}#1\end{flushleft}}}}

	\ifthenelse{\isempty{#2}}{}{
	\textcolor{darkcol} {\cvtext{\textbf{#2}} }
	}

	\ifthenelse{\isempty{#3}}{}{
		\cvtext{{ \textcolor{darkcol} {#3} }}\\
	}

	\cvtext{#4}\\[14pt]
}

%---------------------------------------------------------------------------------------
%	QR CODE
%----------------------------------------------------------------------------------------

% Renders a qrcode image (centered, relative to the parentwidth)
% param 1: percent width, from 0 to 1
\newcommand{\cvqrcode}[1] {
	\begin{center}
		\includegraphics[width={#1}\mpwidth]{qrcode}
	\end{center}
}

%=+=+=+=+=+=+=+=+=+=+=+=+=+=+=+=+=+=+=+=+=+=+=+=+=+=+=+=+=+=+=+=+=+=+=+=+=+=+=+=+
%,,,,,,,,,,,,,,,,,,,,,,,,,,,,,,,,,,,,,,,,,,,,,,,,,,,,,,,,,,,,,,,,,,,,,,,,,,,,,,,,
                       % EDIT AFTER THIS POINT
%''''''''''''''''''''''''''''''''''''''''''''''''''''''''''''''''''''''''''''''''
%=+=+=+=+=+=+=+=+=+=+=+=+=+=+=+=+=+=+=+=+=+=+=+=+=+=+=+=+=+=+=+=+=+=+=+=+=+=+=+=+


%============================================================================%
%
%
%
%	DOCUMENT CONTENT
%
%
%
%============================================================================%
\begin{document}
% \columnratio{0.31}
% \setlength{\columnsep}{2.2em}
% \setlength{\columnseprule}{4pt}
% \colseprulecolor{lightcol}

% \begin{paracol}{2}
% \begin{leftcolumn}
%---------------------------------------------------------------------------------------
%	META IMAGE
%----------------------------------------------------------------------------------------
%\includegraphics[width=\linewidth]{untitled.jpg}	%trimming relative to image size

%---------------------------------------------------------------------------------------
%	TITLE  HEADER
%----------------------------------------------------------------------------------------
\fcolorbox{white}{darkcol}{\begin{minipage}[c][4.5cm][c]{1\mpwidth}
	\begin {center}
		\huge{ \textbf{ \textcolor{white}{  Minh Hai NGUYEN  } } } \\ [-18pt]
		\textcolor{white}{ \rule{0.2\textwidth}{1.25pt} } \\ [3pt]
        \normalsize{
            \iconemail{EnvelopeSquare}{14}{nguyenhai7120qh@gmail.com}{nguyenhai7120qh@gmail.com}{black} 
            \iconemail{Github}{14}{https://mh-nguyen712.github.io}{https://mh-nguyen712.github.io/}{black} 
            \iconemail{Phone}{14}{+33 7 69 60 25 04}{+33 7 69 60 25 04}{black}
        } \\ [3pt]
        \textcolor{white}{ \rule{\textwidth}{1pt} } \\ [3pt]

		\normalsize{ \textcolor{white} {I am a PhD student in MAMBO team at Université de Toulouse (expected graduation: December 2026), advised by \href{https://www.math.univ-toulouse.fr/~weiss/}{Prof. Pierre Weiss} and \href{https://edouardpauwels.fr/}{Prof. Edouard Pauwels}. My research focuses on inverse problems in imaging with applications to microscopy. }}
	\end {center}
\end{minipage}}

%---------------------------------------------------------------------------------------
%	EDUCATION
%----------------------------------------------------------------------------------------
\cvsection{EDUCATION}

\cvevent
	{\textbf{2023 - }}
	{PhD in Applied Mathematics}
	{Toulouse University, France}
	{Advised by \href{https://www.math.univ-toulouse.fr/~weiss/}{Prof. Pierre Weiss} and \href{https://edouardpauwels.fr/}{Prof. Edouard Pauwels}
    \\ Research topics: Inverse Problems in Imaging, applications to Microscopy}
    {}
\vfill\null


\cvevent
	{\textbf{2018 - 2023}}
	{Master of Engineering in Applied Mathematics}
	{INSA Toulouse, France}
	{Major in AI / Data Science \\ {\footnotesize (*Location: 2018-2020 in Vietnam and 2020-2023 in France)}
    }
    {}
\vfill\null

%---------------------------------------------------------------------------------------
%	EXPERIENCE
%----------------------------------------------------------------------------------------
\cvsection{EXPERIENCE}

\cvevent
	{\textbf{Dec. 2022 - Sept. 2023}}
	{Research Engineer}
	{Agenium Space}
	{Correcting micro-vibrations in satellite imaging with pushbroom cameras.}
    {}
\vfill\null

\cvevent
	{\textbf{Jun. 2022 - Sept. 2022}}
	{Research Intern}
	{Institut Mathématiques de Toulouse}
	{Product-convolution neural networks: applications to image deblurring with space-varying blurs.}
    {}
\vfill\null

%---------------------------------------------------------------------------------------
%	PUBLICATION
%----------------------------------------------------------------------------------------
\cvsection{Publications}
\nocite{*}
\printbibliography[heading=none]



%---------------------------------------------------------------------------------------
%	META SKILLS
%----------------------------------------------------------------------------------------
% -------------------------
\cvsection{Language}

\begin{tabular*}{\linewidth}{@{\extracolsep{\fill}} ccccc}
	& \textbf{English:} fluent & \textbf{French:} fluent & \textbf{Vietnamese:} native &
\end{tabular*}
% \vfill\null

\cvsection{Programming Language}

\begin{tabular*}{\linewidth}{@{\extracolsep{\fill}} cccc}
	& \textbf{Python (advanced):} PyTorch (DDP, AMP) & \textbf{C/C++ (intermediate)}  & 
\end{tabular*}
% \vfill\null

\cvsection{Other services}

\textcolor{darkcol}{\textbf{Open source project}} 

\cvevent
	{\href{https://github.com/deepinv/deepinv}{Github Repository}}
	{DeepInverse:}
	{}
	{PyTorch library for solving imaging inverse problems using deep learning}
    {}


\textcolor{darkcol}{\textbf{Teaching}} 
\begin{itemize}
	\item Signal Processing (1st Master level) at INSA Toulouse, in 2024.
	\item Stochastic Optimization and Learning  (1st Master level) at INSA Toulouse / ENSEEIHT, in 2024.

\end{itemize}

\end{document}